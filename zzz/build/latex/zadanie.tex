%% Generated by Sphinx.
\def\sphinxdocclass{report}
\documentclass[letterpaper,10pt,polish]{sphinxmanual}
\ifdefined\pdfpxdimen
   \let\sphinxpxdimen\pdfpxdimen\else\newdimen\sphinxpxdimen
\fi \sphinxpxdimen=.75bp\relax
\ifdefined\pdfimageresolution
    \pdfimageresolution= \numexpr \dimexpr1in\relax/\sphinxpxdimen\relax
\fi
%% let collapsible pdf bookmarks panel have high depth per default
\PassOptionsToPackage{bookmarksdepth=5}{hyperref}

\PassOptionsToPackage{booktabs}{sphinx}
\PassOptionsToPackage{colorrows}{sphinx}

\PassOptionsToPackage{warn}{textcomp}
\usepackage[utf8]{inputenc}
\ifdefined\DeclareUnicodeCharacter
% support both utf8 and utf8x syntaxes
  \ifdefined\DeclareUnicodeCharacterAsOptional
    \def\sphinxDUC#1{\DeclareUnicodeCharacter{"#1}}
  \else
    \let\sphinxDUC\DeclareUnicodeCharacter
  \fi
  \sphinxDUC{00A0}{\nobreakspace}
  \sphinxDUC{2500}{\sphinxunichar{2500}}
  \sphinxDUC{2502}{\sphinxunichar{2502}}
  \sphinxDUC{2514}{\sphinxunichar{2514}}
  \sphinxDUC{251C}{\sphinxunichar{251C}}
  \sphinxDUC{2572}{\textbackslash}
\fi
\usepackage{cmap}
\usepackage[T1]{fontenc}
\usepackage{amsmath,amssymb,amstext}
\usepackage{babel}



\usepackage{tgtermes}
\usepackage{tgheros}
\renewcommand{\ttdefault}{txtt}



\usepackage[Sonny]{fncychap}
\ChNameVar{\Large\normalfont\sffamily}
\ChTitleVar{\Large\normalfont\sffamily}
\usepackage{sphinx}

\fvset{fontsize=auto}
\usepackage{geometry}


% Include hyperref last.
\usepackage{hyperref}
% Fix anchor placement for figures with captions.
\usepackage{hypcap}% it must be loaded after hyperref.
% Set up styles of URL: it should be placed after hyperref.
\urlstyle{same}

\addto\captionspolish{\renewcommand{\contentsname}{Spis treści:}}

\usepackage{sphinxmessages}
\setcounter{tocdepth}{1}



\title{Zadanie}
\date{09 lis 2025}
\release{1}
\author{Filip Gmyrek}
\newcommand{\sphinxlogo}{\vbox{}}
\renewcommand{\releasename}{Wydanie}
\makeindex
\begin{document}

\ifdefined\shorthandoff
  \ifnum\catcode`\=\string=\active\shorthandoff{=}\fi
  \ifnum\catcode`\"=\active\shorthandoff{"}\fi
\fi

\pagestyle{empty}
\sphinxmaketitle
\pagestyle{plain}
\sphinxtableofcontents
\pagestyle{normal}
\phantomsection\label{\detokenize{index::doc}}


\sphinxAtStartPar
Add your content using \sphinxcode{\sphinxupquote{reStructuredText}} syntax. See the
\sphinxhref{https://www.sphinx-doc.org/en/master/usage/restructuredtext/index.html}{reStructuredText}
documentation for details.

\sphinxstepscope


\chapter{Wprowadzenie}
\label{\detokenize{rozdzial1/index:wprowadzenie}}\label{\detokenize{rozdzial1/index::doc}}
\sphinxAtStartPar
Robert Kubica to postać, która na trwałe zapisała się w historii sportu motorowego — nie tylko jako pierwszy Polak w Formule 1, ale także jako symbol wytrwałości, odwagi i pasji. Urodził się 7 grudnia 1984 roku w Krakowie. Od najmłodszych lat fascynowały go samochody, a jego rodzice wspominają, że już jako kilkuletni chłopiec potrafił godzinami oglądać wyścigi i rozpoznawać marki aut po samym dźwięku silnika.

\sphinxAtStartPar
Pierwszym poważnym krokiem w jego karierze był karting. Kubica zaczął ścigać się w wieku zaledwie sześciu lat, a jego determinacja szybko przyniosła efekty.  Zdobył mistrzostwo Polski w 1995 i 1996 roku, a następnie przeniósł się do Włoch \sphinxhyphen{} centrum europejskiego kartingu. Tam ścigał się z przyszłymi gwiazdami Formuły 1,  takimi jak Felipe Massa czy Fernando Alonso.

\sphinxAtStartPar
Kolejne lata przyniosły szybki awans. W 2002 roku Kubica zdobył Europejski Puchar Formuły Renault, co otworzyło mu drzwi do wyższych kategorii wyścigowych. Wkrótce pojawił się w Formule 3 Euro Series i Formule BMW, gdzie odnosił kolejne sukcesy, zdobywając reputację jednego z najbardziej utalentowanych młodych kierowców w Europie.

\sphinxAtStartPar
Kubica słynął z precyzyjnego stylu jazdy i niezwykłej umiejętności analizy toru. Jego inżynierowie często podkreślali, że potrafił opisać zachowanie samochodu z dokładnością do pojedynczego zakrętu. Był perfekcjonistą, który dążył do doskonałości w każdym aspekcie.

\sphinxAtStartPar
Dzięki temu podejściu szybko zyskał uznanie w świecie motorsportu i został zauważony przez zespoły Formuły 1. W 2005 roku został kierowcą testowym BMW Sauber, a już rok później zadebiutował w wyścigu Grand Prix. Dla Polski był to historyczny moment \sphinxhyphen{} po raz pierwszy nasz kraj miał swojego reprezentanta w najbardziej prestiżowej serii wyścigowej świata.

\sphinxAtStartPar
Kubica stał się nie tylko sportowcem, ale również inspiracją dla całego pokolenia młodychludzi, którzy dzięki niemu zaczęli interesować się kartingiem i wyścigami samochodowymi.

\sphinxstepscope


\chapter{Kariera w F1}
\label{\detokenize{rozdzial2/index:kariera-w-f1}}\label{\detokenize{rozdzial2/index::doc}}
\sphinxAtStartPar
Kubica zadebiutował w F1 w 2006 roku w zespole BMW Sauber. Jego dynamiczna jazda i szybkie czasy okrążeń szybko przyniosły mu uznanie w stawce. W ciągu kolejnych sezonów zdobywał miejsca na podium, imponując nie tylko prędkością, ale i strategicznym myśleniem.


\section{Tabela wyników w latach 2006\sphinxhyphen{}2010}
\label{\detokenize{rozdzial2/index:tabela-wynikow-w-latach-2006-2010}}

\begin{savenotes}\sphinxattablestart
\sphinxthistablewithglobalstyle
\centering
\begin{tabulary}{\linewidth}[t]{TTTT}
\sphinxtoprule
\sphinxstyletheadfamily 
\sphinxAtStartPar
Sezon
&\sphinxstyletheadfamily 
\sphinxAtStartPar
Zespół
&\sphinxstyletheadfamily 
\sphinxAtStartPar
Miejsce w klasyfikacji
kierowców
&\sphinxstyletheadfamily 
\sphinxAtStartPar
Najlepszy wynik
w wyścigu
\\
\sphinxmidrule
\sphinxtableatstartofbodyhook
\sphinxAtStartPar
2006
&
\sphinxAtStartPar
BMW Sauber
&
\sphinxAtStartPar
16
&
\sphinxAtStartPar
P3 GP Włoch
\\
\sphinxhline
\sphinxAtStartPar
2007
&
\sphinxAtStartPar
BMW Sauber
&
\sphinxAtStartPar
6
&
\sphinxAtStartPar
P4 GP Hiszpanii, Francji i UK
\\
\sphinxhline
\sphinxAtStartPar
2008
&
\sphinxAtStartPar
BMW Sauber
&
\sphinxAtStartPar
4
&
\sphinxAtStartPar
P1 GP Kanady
\\
\sphinxhline
\sphinxAtStartPar
2009
&
\sphinxAtStartPar
BMW Sauber
&
\sphinxAtStartPar
14
&
\sphinxAtStartPar
P2 GP Brazylii
\\
\sphinxhline
\sphinxAtStartPar
2010
&
\sphinxAtStartPar
Renault
&
\sphinxAtStartPar
8
&
\sphinxAtStartPar
P2 GP Australii
\\
\sphinxbottomrule
\end{tabulary}
\sphinxtableafterendhook\par
\sphinxattableend\end{savenotes}

\sphinxAtStartPar
Kubica zasłynął przede wszystkim dzięki wygranej w Grand Prix Kanady 2008 odnosząc swoje pierwsze i ostatnie zwysicięstwo w F1.

\sphinxAtStartPar
Wypadek w 2011 roku podczas rajdu Ronde di Andora, spowodował poważne obrażenia ręki, co wykluczyło go z wyścigów F1 na kilka lat. Rehailitacja była długa i trudna, a powrót wydawał się nierealny.

\begin{figure}[htbp]
\centering
\capstart

\noindent\sphinxincludegraphics{{obraz}.jpg}
\caption{Kubica  w bolidzie BMW Sauber}\label{\detokenize{rozdzial2/index:id1}}\end{figure}

\sphinxstepscope


\chapter{Powrót do F1 i życie po wyścigach}
\label{\detokenize{rozdzial3/index:powrot-do-f1-i-zycie-po-wyscigach}}\label{\detokenize{rozdzial3/index::doc}}
\sphinxAtStartPar
Po długiej i wymagającej rehabilitacji Robert Kubica w 2019 roku ponownia zasiadł za kierownicą bolidu Formuły 1, reprezentując zespół Williams. Jego powrót do stawki był jednym z najbardziej emocjonujących momentów w historii współczesnego motorsportu. Wielu ekspertów uważało, że po tak poważnym wypadku nigdy nie będzie w stanie wrócić do rywalizacji na najwyższym poziomie, jednak Kubica po raz kolejny udowodnił, że dla niego nie ma rzeczy niemożliwych.

\sphinxAtStartPar
Poimmo że bolid Williamsa nie był w pełni konkurencyjny, sam fakt powrotu polaka do ścigania był ogromnym sukcesem. Jego determinacja, koncentracja i doświadczenie sprawiły, że wzbudził podziw zarówno wśród rywali, jak i kibiców na całym świecie.

\sphinxAtStartPar
Sezon 2019 był dla niego trudny, ale jednocześnie symboliczny. Wykazał się ogromną wytrwałością, pracując z inżynierami nad rozwojem bolidu i wspierając młodszych zawodników. Dzięki niemu wiele kibiców w Polsce ponownie zainteresowało się F1, a jego historia stała się przykładem niezwykłej siły charakteru i pasji.

\sphinxAtStartPar
Po zakończeniu regularnych startów w F1 Kubica kontynuował karierę w rajdach oraz seriach długodystansowych, a także pełnił funkcję kierowcy testowego i rezerwowego w zespołach F1 \sphinxhyphen{} Alfa Romeo i później Ferrari. Jego doświadczenie techniczne i analityczne podejście do pracy z bolidem czyniły go niezwykle cennym członkiem zespołu.

\sphinxAtStartPar
Poza torem Robert angażuje się w promocję motorsportu w Polsce, wspiera młodych kierowców i regularnie uczestniczy w projektach popularyzujących bezpieczeństwo w ruchu drogowym. Jest również komentatorem i ekspertem, dzielącym się wiedzą z kibicami. Jego historia inspiruje kolejne pokolenia do wiary w siebie i w to, że nawet po najtrudniejszych chwilach można wrócić silniejszym.


\section{Dodatkowe aktywności}
\label{\detokenize{rozdzial3/index:dodatkowe-aktywnosci}}\begin{itemize}
\item {} 
\sphinxAtStartPar
Kierowca testowy i rezerwowy w zespołach F1

\item {} 
\sphinxAtStartPar
Udział w rajdach WRC i innych imprezach motoryzacyjnych

\item {} 
\sphinxAtStartPar
Promowanie sportów motorowych w Polsce i za granicą

\item {} 
\sphinxAtStartPar
Wsparcie młodych talentów w kartingu i wyścigach samochodowych

\item {} 
\sphinxAtStartPar
Współpraca z mediami i publikacje o tematyce motorsportowej

\end{itemize}

\sphinxstepscope


\chapter{Dziedzictwo i wpływ na polski motorsport}
\label{\detokenize{rozdzial4/index:dziedzictwo-i-wplyw-na-polski-motorsport}}\label{\detokenize{rozdzial4/index::doc}}
\sphinxAtStartPar
Robert Kubica pozostaje niekwestionowaną ikoną polskiego sportu motorowego. Jego osiągnięcia zapisały się złotymi zgłoskami w historii nie tylko Formuły 1, lecz także całego polskiego motorsportu. Dzięki jego sukcesom miliony Polaków zaczęły interesować się wyścigami, a transmisje F1 po raz pierwszy przyciągnęły szeroką publiczność w kraju.

\sphinxAtStartPar
Kubica swoim przykładem udowodnił, że nawet pochodząc z kraju bez bogatych tradycji w wyścigach, można przebić się na sam szczyt światowej rywalizacji. Stał się wzorem determinacji, dyscypliny i profesjonalizmu. Jego podejście do pracy \sphinxhyphen{} chłodne, analityczne i jednocześnie pełne pasji \sphinxhyphen{} zainspirowało nie tylko młodych kierowców, ale też całą społeczność sportową.

\sphinxAtStartPar
Jego sukcesy w Formule 1, rajdach i wyścigach długodystansowych sprawiły, że w Polsce zaczęły powstawać nowe szkoły kartingowe, akademie kierowców i inicjatywy wspierające młodych zawodników. Wielu współczesnych polskich kierowców, takich jak Kacper Sztuka czy Tymek Kucharczyk, otwarcie przyznaje, że to właśnie Kubica był dla nich inspiracją do rozpoczęcia kariery.

\sphinxAtStartPar
Poza torem Kubica odgrywa także ważną rolę jako ambasador sportów motorowych. Często uczestniczy w wydarzeniach promujących bezpieczeństwo na drogach i rozwój motorsportu w Polsce. Jego nazwisko jest synonimem profesjonalizmu, a jego postawa \sphinxhyphen{} dowodem na to, że wytrwałość może pokonać wszelkie przeciwności losu.


\section{Wpływ Kubicy na motorsport w Polsce}
\label{\detokenize{rozdzial4/index:wplyw-kubicy-na-motorsport-w-polsce}}\begin{itemize}
\item {} 
\sphinxAtStartPar
Inspiracja dla młodych kierowców i wzrost liczby szkół kartingowych

\item {} 
\sphinxAtStartPar
Promowanie wyścigów samochodowych w mediach krajowych

\item {} 
\sphinxAtStartPar
Wsparcie dla polskich zawodników startujących w wyścigach międzynarodowych

\item {} 
\sphinxAtStartPar
Podniesienie prestiżu Polski na mapie sportów motorowych

\item {} 
\sphinxAtStartPar
Rozwój infrastruktury i programów szkoleniowych dla młodych kierowców

\end{itemize}

\sphinxstepscope


\chapter{Zwycięstwa w wyścigach długodystansowych i Le Mans}
\label{\detokenize{rozdzial5/index:zwyciestwa-w-wyscigach-dlugodystansowych-i-le-mans}}\label{\detokenize{rozdzial5/index::doc}}
\sphinxAtStartPar
Po powrocie do wyścigów długodystansowych Robert Kubica szybko udowodnił, że jego talent i determinacja nie zanikły mimo trudnych lat po wypadku. W serii endurance odnalazł nowe wyzwanie \sphinxhyphen{} nie polegające wyłącznie na czystej prędkości, lecz także na strategii, współpracy z zespołem i utrzymaniu koncentracji przez wiele godzin jazdy. Starty w prestiżowych zawodach, takich jak 24\sphinxhyphen{}godzinny wyścig Le Mans, pokazały jego niezwykłą zdolność adaptacji do innego stylu rywalizacji.

\sphinxAtStartPar
Dzięki swojemu doświadczeniu z Formuły 1, Kubica znakomicie odnajduje się w realiach wyścigów WEC (World Endurance Championship) i ELMS (European Le Mans Series). Jego techniczna wiedza pomagała zespołom w precyzyjnym dostrajaniu samochodów, a jego regularność na torze wielokrotnie dawała przewagę w kluczowych momentach rywalizacji.

\sphinxAtStartPar
W 2021 roku odniósł jedno ze swoich największych osiągnięć w wyścigach endurance, zdobywając tytuł mistrzowski w serii ELMS w klasie LMP2 z zespołem Team WRT. Jego występ w Le Mans tego samego roku, mimo dramatycznego finału z awarią na ostatnim okrążeniu, pokazał ogromne serce do walki i odporność psychiczną.


\section{Najważniejsze zwycięstwa i osiągnięcia}
\label{\detokenize{rozdzial5/index:najwazniejsze-zwyciestwa-i-osiagniecia}}\begin{enumerate}
\sphinxsetlistlabels{\arabic}{enumi}{enumii}{}{)}%
\item {} 
\sphinxAtStartPar
24h Le Mans \sphinxhyphen{} wygrana w 2025 roku

\item {} 
\sphinxAtStartPar
Zwycięstwa w wyścigach FIA WEC \sphinxhyphen{} w klasie LMP2

\item {} 
\sphinxAtStartPar
Triumfy w wyścigach długodystansowych na torach Silverstone i Spa\sphinxhyphen{}Francorchamps

\item {} 
\sphinxAtStartPar
Wygrana w European Le Mans Series w 2021 z zespołem WRT

\end{enumerate}

\sphinxAtStartPar
Kubica udowodnił, że jego umiejętności nie ograniczają się tylko do Formuły 1. Jego wszechstronność, precyzja i nieustępliwość czynią go jednym z najbardziej szanowanych kierowców w świecie wyścigów długodystansowych. Jego kariera to dowód, że prawdziwa pasja i profesjonalizm potrafią przekraczać granice dyscyplin.



\renewcommand{\indexname}{Indeks}
\printindex
\end{document}